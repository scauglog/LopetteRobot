\paragraph{Optimisation du code:}
toujours dans le but d'obtenir un binaire aussi petit que possible,
nous avons utilisés des entiers codés sur 16 et 8 bits en
priorité. nous avons éviter d'utiliser les fonctions mallocs et
free. Nous avons utiliser des commandes préprocesseurs ne compiler 
le code que pour l'architecture pour laquel il était destiné.

\paragraph{Interruption:}
Nous avons utilisé des interruptions pour bufferiser les caractères
venant du port série. C'est également durant l'interruption que l'on
lance le parsage du buffer lorsque la commande est terminé.


\paragraph{Parsing:} 

Le système parse le buffer RX lorsque la différence entre le nombre
d'octet lu et la taille du paquet indiqué est égale à 0. le système
lit ensuite l'octet contenant la commande désirée. Ce sera ensuite à
elle d'aller lire les octets qui l'interesse dans le buffer RX.

\paragraph{Description rapide des commandes:}

\begin{description}
\item[GetCaps :] La fonction GetCaps va lire le nombre de pin (stocké dans une variable d'environnement) puis écris la trame de la réponse dans le buffer WX.
\item[Reset:] La fonction Reset appelle la fonction "InitPins" qui remet l'état, la valeur et le type de chaque pin dans l'état initial.
\item[Ping :] Cette fonction renvoie le version du protocol installée sur la Arduino. Celle-ci est écrite en dur dans le code.
\item[SetType :] Cette fonction parcours le masque de la trame contenu dans RX et met à jour le type des pins en fonction des valeurs lues.
\item[GetType :] Cette fonction écris dans WX les types des pins demandé en parcourant le tableau de pin.
\end{description}

\paragraph{Algorithme de SetType}
voici l'algorithme recuperer un des bits du type.



\begin{verbatim}
 if(rx[3+i/8]&(1<<(7-(i%8)))){
 uint8_t val2=rx[3+(PINSNUMBER+npa*3+1)/8]&(0b10000000>>((PINSNUMBER+3*npa+1)%8));
 }        
\end{verbatim}

La condition permet reperer les bits concerné par la commande.

\begin{itemize}
\item $3 + (PINSNUMBER + npa*3+1)/8$ permet de recuperer le numéro de l'octet qui code 
un bit du type de la pin.
\item le premier trois permet de se placer au debut masque.
\item PINSNUMBER permet de sauter le masque.
\item npa represente le nombre de pin déjà traité.
\item on le multiplie par trois qui correspond au codage du type.
\item on divise par 8 pour avoir le nombre de bit.
\end{itemize}

Nous avons maintenant selectionner le bon octet, il reste a récuperer le bon bit, pour cela nous utilisons l'operateur & et un décalage vers la droite.
le "+1" correspond au bit du type recherché.


%% $
%% \begin{array}{|c|c|c|c|c|c|c|c|}
%% \hline
%% 1&0&1&0&0&0&0&0\\
%% \hline
%% \end{array}
%% ~~
%% \begin{array}{|c|c|c|c|c|c|c|c|}
%% \hline
%% 0&0&0&0&0&0&0&0\\
%% \hline
%% \end{array}
%% ~~
%% \begin{array}{|c|c|c|c|c|c|c|c|}
%% \hline
%% 0&0&0&0&0&0&1&1\\
%% \hline
%% \end{array}
%% ~~
%% \begin{array}{|c|c|c|c|c|c|c|c|}
%% \hline
%% 0&0&0&0&1&0&1&0\\
%% \hline
%% \end{array}
%% $

