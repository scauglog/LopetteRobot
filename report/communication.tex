Dans cette partie nous allons décrire notre approche vis-à-vis de la 
communication série via l'UART. 

\subsection{Bufferisation}
Décrivons ce qu'il se passe lors de la réception d'un octet par l'Atmega8. 
Dans un premier temps, une interruption est levé (USART\_RXC\_vect). Nous 
utilisons alors une commande préprocesseur \textit{ISR}, fournie par avr, 
pour redéfinir le comportement de la carte lors de la-dite interruption.
Nous récupérons alors l'octet reçu et le stockons dans un buffer circulaire
de taille prédéfinie. Cette taille a été choisie de sorte qu'elle puisse 
accepter n'importe quelle commande pour notre carte. Ayant un nombre limité
de pin sur notre carte, nous limitons ainsi la taille des données. En 
effet, la commande la plus longue est l'écriture de PWM16 sur l'ensemble
des pins. La taille totale de la trame est alors (en octet) : \\
$$
\begin{array}{ccccccccccc}
1     &+&   2&+&   3&+&  23*2&+&       1&=&53\\
Header&+&Size&+&Mask&+&Values&+&Checksum& &\\
\end{array}
$$
~\\

Dans une première phase de test, ce buffer était lu dans la boucle principal
du main et une réponse était élaborer selon le premier octet. Nous allumions une
LED en appuyant sur une touche et l'éteignant en appuyant sur une autre.
Ce comportement à néanmoins changé lorsque nous avons voulu faire à la fois
du PWM logiciel en même temps que la communication sur l'UART.
Ce PWM logiciel a été implémenté et semblait nécessiter un thread pour opéré.
Nous avons alors pensé en créer en plus du main avec des bibliothèques de thread.
Mais outre la perdre inutile d'espace mémoire, nous perdions surtout
l'aspect temps réel de la communication; ce que nous tenons à conserver.
Après quelques essais avec ces threads, nous avons donc choisi d'attaquer le
problème autrement. Nous utilisons alors l'interruption pour lancer le parse
de la commande et son éxécution.\\

Cette solution a tout d'abord été émulée avant d'être envoyé sur la carte 
Arduino.

\subsection{Émulation}
Pour émuler les interruptions du processeur nous avons choisi d'utiliser 
la bibliothèque de thread standart. Le code faisant la bufferisation
des caractères a alors été factorisée dans la fonction
\textit{initiateParse} du \textit{main.c}. Le PWM et la réponse à une commande
fonctionnaient alors convenablement sur l'émulateur.\\

Néanmoins, lorsque l'on a téléversé le code sur l'Atmega8, le résultat était
différent. Après une longue phase de debug qui ne s'est jamais vraiment terminée,
nous avons compris que le problème provenait d'un conflit temporel entre
la réception des octets et la durée de l'interruption. Nous avons trouvé de 
nombreuses options à passer à ces interruptions telles que: ISR\_NOBLOCK et 
ISR\_BLOCK. Ces deux options en particulier permettent d'activer ou non les 
interruptions lorsque cette interruption est en cours d'éxécution. Par exemple,
avec ISR\_NOBLOCK on peut se retrouver avec une pile d'interruption assez importante
alors que l'autre l'empêche.\\

Nous nous sommes essentiellement penché sur ces deux options:
\begin{description}
\item[ISR\_NOBLOCK :]
  Nous avons essayer, sans succès, de remodeler le code de sorte que la pile
  d'interruption n'empêche pas le bon remplissage du buffer.
\item[ISR\_BLOCK :]
  Nous avons cherché s'il existait une file d'interruption pour pouvoir accéder
  aux octets qui n'ont pas été traité. Cette file ne semble pas exister.
\end{description}

\subsection{Une autre solution}
Ayant perdu trop de temps sur cet aspect, nous n'avons pas pris le temps d'essayer
d'utiliser des ``Timer Interrupt'' (TIMER1\_OVF\_vect) pour régler le PWM.
Cela semble être, a posteriori, la meilleure méthode car elle libère le main pour 
la gestion des commandes car la gestion de cet interruption est peu coûteuse en terme
de temps : mettre à 1 où 0 les bons PORTS aux moments.

%\paragraph{Baud rate :}
%Nous utilisons un baud rate fixé à 115200, ce qui a posé problème au
%début car la formules fournies par la documentation nous donnait une
%valeur fausse (le machine nous renvoyant la partie entière du
%résultat, alors que le micro processeur attendait la valeur arrondie
%pour ce baudrate la).
