\subsection{Firmware, Driver et Émulateur}
Dans ce rapport, nous désignons par Firmware l'application côté Arduino. 
Le Driver est l'application côté Raspberry, aussi appelé client. Et l'Émulateur
est une application compilée pour PC afin de simuler le comportement de
l'Arduino.\\

La structure du projet a été pensée pour pouvoir compiler un même code C
(ou presque) pour le Firmware et l'Émulateur. On retrouve ce code dans le
dossier \textit{src} à la racine du projet. Pour construire les binaires nous avons 
un dossier \textit{build} dans lequel nous avons mis un script qui permet de 
téléverser semi-automatiquement. Nous avons
compris la nécessité d'un tel script après avoir rapidement écrasé le bootloader
de la carte; Pour éviter de lancer avrdude en ayant oublié d'allégé le binaire
avec avr-objcopy. Une amélioration de ce script aurait été de tester la taille 
du binaire avant de l'envoyer sur la carte. Mais nous n'avons pas approché
cette limite.\\

Le Driver n'a pas été implémenté. Seul un client, codé à partir d'une bibliothèque
fourni lors d'un TP précédent à été utilisé pour envoyer des requêtes précises.
Son utilité est à relativiser car nous utilisions surtout du shell :
\begin{center}
  \$ echo '\textbackslash x01\textbackslash x00\textbackslash x03abc\textbackslash x44' | ./bin/LopetteRobot
\end{center}

\subsection{Organisation des sources}
Les sources sont découpées en plusieurs modules autour du \textit{main.c}. Notre
objectif a été de séparer les commandes systèmes propres à avr du traitement des commandes
et de la communication. Nous avons donc créé 3 modules dans ce sens :
\begin{itemize}
\item Parse 
  { 
    \it 
    Lancé dans la boucle général du Firmware, son rôle est de lancer
    les commandes avr appropriées en fonction de la commande texte reçue. Son rôle s'étend 
    donc à décomposer le flux de bits reçu en plusieurs char qui seront envoyés en tant 
    qu'option pour les commandes. Elle devrait aussi contrôler la cohérence du flux entrant
    en vérifiant par exemple sa checksum.
  }
\item BindingAVR
  {
    \it
    Interface entre l'aspect pûrement logiciel du projet et les appels systèmes. Ainsi,
    chaque fonction qu'il contient fournira une version différente selon les options de 
    compilation. Un exemple typique est l'écriture sur l'UART qui est traduit par un 
    simple fprintf sur la sortie standard.
  }
\item Commande
  {
    \it
    Liste l'ensemble des fonctions du protocole qui pourront être appelées depuis le fichier 
    Parse. Il devra alors utiliser les fonctions fournies par l'interface BindingAVR.
  }
\end{itemize}

Un \textit{common.h} a de plus été ajouté à cette hiérarchie afin factorisation l'inclusion des
différents modules et la déclaration de variables globales selon la type de compilation
choisie.\\


\subsection{Compilation}
Comme expliqué dans le \textit{readme.md} à la racine du projet, un flag de compilation doit
être donné à cmake. Depuis le répertoire build, utilisez l'option mode pour définir le type
de compilation:
\begin{center} cmake .. -DMODE=ARDUINO \&\& make\end{center}

Toute autre valeur compilera pour PC:
\begin{center} cmake .. -DMODE=PC \&\& make\end{center}
